As before, let a point $P(t)$ on $\mathcal{E}$ be parametrized as $P(t)=[a\cos t,b\sin t]$. Define the {\em signed area} of a curve $\gamma$ as:

\begin{equation}
\mathcal{A}_\gamma=\frac{1}{2}\int_{\gamma}(x{dy}-y{dx}).
\label{eqn:area}
\end{equation}

%The support function of the ellipse is $h(t)=\sqrt{a^2\cos^2t+b^2\sin^2 t}$.

Referring to Figure~\ref{fig:evolutoid}, the $\theta$-evolutoid is the envelope of lines passing through $P(t)$ rotated with respect to the tangent vector $P'(t)$ by $\theta$. Its coordinates can be derived explicitly as

\begin{align*}
x_{\theta}(t)=& a\cos^{2}{\theta}\cos{t}
  + \frac {c^2 \sin^2  \theta 
		\cos^3   t  
	 }{a} - \frac {\sin t \sin   \theta
	  \cos  \theta    (   b^2\cos^2 t
		 +{a}^{2}   \sin^2 t )}{b} 
 \\
y_{\theta}(t)=&a\sin   \theta
\cos  \theta  \cos   t - \frac {c^2\sin{\theta} \cos^{2}t \left( b\cos
		t \cos  \theta  -a\sin  \theta
	  \sin t \right) }{ab}  \\
  +& \frac {
		\sin t   \left( b^2\cos^2   \theta 
	  -c^2 \sin^2  \theta  
		 \right)}{b}
%
\end{align*}

\noindent with $c^2=a^2-b^2$.

Let $\theta_0=\tan^{-1}\left(\frac{2ab}{{3}c^2}\right)$.

\begin{remark}
 The parametrization $[x_\theta(t), y_\theta(t)]$ of the $\theta$-evolutoid will   have 4, 2, or 0 singularities if
$\theta\in(\theta_0,\pi-\theta_0)$, $\theta\in\{\theta_0,\pi-\theta_0\}$, or $\theta\notin[\theta_0,\pi-\theta_0]$, respectively. Moreover, the ${\theta_0}$-evolutoid is singular at $t_1=\frac{3\pi}{4}$ and
	$t_2=  \frac{7\pi}{4}$.
	Here, a  point $t=t_0$ is called singular when $x'_\theta(t_0)=y'_\theta(t_0)=0$.
	These points correspond to real cusps of the curve.  
	In the implicit form,   a  $\theta$-evolutoid is given by $f^{-1}(0)$, and $f$ being a polynomial   function of degree 12.
\end{remark}

\begin{proposition}
The signed area $S_\theta$ of the 
$\theta$-evolutoid  of the ellipse $\mathcal E$ is given by:
%
\[S_{\theta}=\pi a b \cos^2\theta -\frac{  3 c^4\pi}{8 ab} \sin^2\theta\]
% \label{prop:aelipse}
\end{proposition}
\begin{proof}
Direct integration of Equation~\ref{eqn:area}.
\end{proof}

\begin{proposition}\label{prop:areapc}
The areas $A_p$ and $A_c$ of $\mathcal{E}_p$ and  $\mathcal{E}_c$ are given by: 

\begin{align}
A_p=&\frac{\pi}{2}\, \left( {a}^{2}+{b}^{2}+   x_0 ^{2}+  y_0^{2} \right) \label{eqn:ap} \\
A_c=&\frac{\pi}{2}\, \left( (a-b)^2 +   x_0 ^{2}+  y_0^{2} \right) \nonumber
\end{align}
where $M = [x_0,y_0]$. 
\end{proposition}


\begin{proof} Consider the ellipse parametrized by $P(t)=[a\cos t,b\sin t]$. Then it follows that
{\small 
\[\aligned
	\mathcal{E}_p(t)&=  \left[\frac{   a^2 x_0\sin^2 t -ab y_0\cos t\sin t+ab^2\cos t }{   b^2\cos ^2t+a^2\sin ^2t },  \frac{ b^2y_0\cos^2t -abx_0\cos t \sin t +a^2b\sin  t}{   b^2 \cos ^2t +a^2 \sin^2 t}\right]\\
	\mathcal{E}_c(t)&=\left[ \frac{\cos t( b^2x_0\cos t  + a b y_0 \sin t + a c^2\sin^2{t})}{ b^2 \cos ^2t +a^2 \sin^2 t},
	\frac{ \sin t( a b x_0 \cos t  - a^2 y_0\sin t   + b c^2\cos^2{t})  }{ b^2 \cos ^2t +a^2 \sin^2 t}\right]
	%\mathcal{E}_c(t)=&\left[ \frac{b^2x_0\cos ^2t  + \cos t\sin t %(aby_0+ac^2\sin{t})}{ b^2 \cos ^2t +a^2 \sin^2 t},
%	\frac{  a^2 y_0\sin ^2t + \cos t\sin %t(abx_0-b c^2\cos{t})  }{ b^2 \cos ^2t +a^2 \sin^2 t}\right]
	\endaligned \]
	}
 
Compute the above areas with Equation~\eqref{eqn:area}. The integrand will be a ratio of trigonometric polynomials. Evaluate the integrals by using classical residue theory \cite{ahlfors1979-complex}. Algebraic manipulation yields the claim.
\end{proof}

\noindent Note: formulas in \eqref{eqn:ap} and later are consistent with Steiner's result that the area of the pedal curve of $M$ is the sum of the area for $M=K$ and a term proportional to the square of $|MK|$ \cite[p.~47]{steiner1838}.

\begin{corollary}
$A_p-A_c=A$.
\label{cor:area-diff}
\end{corollary}

\noindent As shown in Section~\ref{sec:epilogue}, the above holds holds for all smooth curves.

The rotated pedal curve $\mathcal{E}_{\theta}$ is given by
$[X_\theta/\Delta,Y_\theta/\Delta]$ where:
{\small 
\begin{align*}
 X_\theta &=  - 
  \left(     {a}^{2}  \sin^2  t   
\cos^2\theta +\frac{1}{2} a b\sin \left( 2\,\theta
 \right)  \sin \left( 2\,t \right) +  b^2  \cos^2 t 
  \sin^2\theta  
 \right) {x_0} \\
&+ \frac{1}{2} \left(  (b^2
  \cos^2 t   - {a}^{2} \sin^2t)\sin \left( 2\,\theta \right)
  +ab\sin
 \left( 2\,t \right) \cos \left( 2\,\theta \right)   \right)  y_0
 \\
 &+ \frac{b}{4}(c^2 \cos t \sin(2t) + 2 a^2 \sin{ t} )\sin(2\theta) - a \cos t (b^2\cos^2\theta + c^2\, \sin^2 t\sin^2\theta) 
 \\
 Y_\theta&= \frac{1}{2} \left( (  b^2  \cos^2t -   a^2\sin ^2{t} )  \sin(2\theta)+ a b  \sin (2t)  \cos(2  \theta)       \right)  x_0\\
 &+ \left(  (    a^2 \sin ^2{t}-b^2\cos^2t) \cos^2\theta   +\frac{1}{2} a b  \sin(2\theta)     \sin(2 t)  - a^2  \sin ^2{t}\right)  y_0\\
 &+  \frac{c^2}{4}  \left(2b \cos t\sin^2\theta+ a  \sin {t}   \sin(2\theta)    \right)  \sin(2 t) 
  -\frac{ab}{2}\left(2a \cos^2\theta  \sin {t}   +b \cos{t}  \sin(2\theta)  \right)
   \\
 \Delta&={b}^{2}   \cos^2 t +{a}^{2}\sin^2t
\end{align*}
}

\begin{proposition}
The area $A_\theta$ of the rotated pedal curve is given by

%\aligned A(P_M)=&\frac{\pi}{2}\, \left( {a}^{2}+{b}^{2}+   x_0 ^{2}+  y_0^{2} \right)\\
\[ A_\theta=\frac{\pi}{2}\, \left(     {a}^{2}+b^2-2\,ab\sin^2\theta + x_0^{2}+y_0^{2} \right) \]
\end{proposition}

\begin{proof}
Similar to Proposition \ref{prop:areapc}.
\end{proof}

\begin{corollary}
$A_p-A_\theta=A\sin^2\theta$.
\end{corollary}

\begin{proposition}
\label{prop:interpol}
The area $A_\mu$ of $\mathcal{E}_{\mu}=(1-\mu)\mathcal{E}_p+\mu \mathcal{E}_c$ is given by:

\begin{align*} A_{\mu}=& \left(2\,{a}^{2}-3\,ab+2\,b^{2}+2\,x_0^{2}+2\,y_0^{2} \right) \pi\,{\mu}^2 -2\,\left( (a-b)^2  +x_0^{2}
+y_0^{2} \right) \pi\,\mu \\
+&\frac{1}{2}\left( a^2+b^2 +x_0^{2}+y_0^{2} \right) \pi\\
  =&(1-2\mu)[(1-\mu) A_p-\mu A_c]+\mu(1-\mu)A. 
\end{align*}

 
\end{proposition}

\begin{proof}
Similar to Proposition \ref{prop:areapc}. 
\end{proof}